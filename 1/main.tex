\documentclass{article}
\usepackage{graphicx}
\usepackage{amsmath}
\usepackage{array}
\usepackage{fancyhdr}
\usepackage{amssymb}
\usepackage[shortlabels]{enumitem}

\pagestyle{fancy}
\fancyhead[L]{Banghao Chi}
\fancyhead[C]{Homework 1}
\fancyhead[R]{25th Jan}

\fancyfoot[C]{\thepage}

\renewcommand{\headrulewidth}{0.5pt}
\renewcommand{\footrulewidth}{0.5pt}

\begin{document}

\section*{Exercise 1}

Consider the group $(G,\star)$, where $G=\{-1,0,1\}$ and the group operation is defined by
\[
a \star b = \begin{cases}
a + b + 1, & \text{if } a + b \leq 0, \\
a + b - 2, & \text{else.}
\end{cases}
\]

(a) Write the following multiplication table for this group (in the row corresponding to an element $a \in G$ and the column corresponding to an element $b \in G$ one should write the element $a \star b$):
\[
\begin{array}{c|ccc}
& -1 & 0 & 1 \\
\hline
-1 & & & \\
0 & & & \\
1 & & & 
\end{array}
\]

(b) What is the identity element for this group?

(c) What is the inverse operation?

(d) Verify an example of the associativity property of $G$.\\

\textbf{Solution:}\\

(a)

\[
\begin{array}{c|ccc}
& -1 & 0 & 1 \\
\hline
-1 & -1 & 0 & 1\\
0 & 0 & 1 & -1\\
1 & 1 & -1 & 0
\end{array}
\]

(b) From the above table, the identity element is -1. \\

(c) In order to find the inverse operation, we need to find formulas that give us identity element($a \star a^{-1} = a^{-1} \star a = e$), which in this case $e = -1$.

From the table, we can see that:

\begin{align*}
-1 \star -1 &= -1 \\
0 \star 1 &= -1 \\
1 \star 0 &= -1
\end{align*}

Therefore, the inverse function is:
\[
a^{-1} = \begin{cases}
-1, & \text{if } a = -1, \\
1, & \text{if } a = 0, \\
0, & \text{if } a = 1.
\end{cases}
\]

(d) For example:
\begin{align*}
(-1 \star 0) \star 1 &= 0 \star 1 = -1 \\
-1 \star (0 \star 1) &= -1 \star -1 = -1
\end{align*}

Therefore, the associativity property is satisfied.

\newpage

\section*{Exercise 2}

Let $G$ be a group.
\begin{enumerate}[(a)]
\item Let $a,b \in G$. Show that there exists a unique element $x \in G$ such that $ax=b$.
\item Prove the cancellation law: if $ab=ac$ for some $a,b,c \in G$, then $b=c$.
\end{enumerate}

\textbf{Solution:}\\

(a) Since $G$ is a group, there exists its inverse element $a^{-1} \in G$ such that $a \star a^{-1} = a^{-1} \star a = e$. Therefore, we can construct the x by:

$$x = a^{-1} \star b$$

Now, we will prove the existence and the uniqueness of x.

\begin{enumerate}[(i)]
\item Existence:

In order to verify the existence of x, we need to plug it into the equation $ax=b$ and see if the equation holds.

$$a \star x = a \star (a^{-1} \star b) = (a \star a^{-1}) \star b = e \star b = b$$

Therefore, x exists.

\item Uniqueness:

In order to verify the uniqueness of x, we need to prove that if there exists another element $x' \in G$ such that $ax'=b$, then $x=x'$.

\begin{align*}
x & = a^{-1} \star b \\
& = a^{-1} \star (a \star x') \\
& = (a^{-1} \star a) \star x' \\
& = e \star x' \\
& = x'
\end{align*}

Therefore, x is unique.

\end{enumerate}

\newpage

\section*{Exercise 3}

Consider the set $G=\{e^{2\pi i k/5} : k=0,1,2,3,4\}$ with the operation of usual multiplication $a \cdot b := ab$.
\begin{enumerate}[(a)]
\item Show that $(G,\cdot)$ is an Abelian group.
\item What subgroups does this group have?
\end{enumerate}

\textbf{Solution:}\\

(a) In order to show that $(G,\cdot)$ is an Abelian group, we need to show that it is a group(closed, associative, identity, inverse) and that it is Abelian(commutative).

\begin{enumerate}[(i)]
    \item \textbf{Closure:}\\
    Take any two elements in \(G\). They must be of the form
    \[
    e^{2\pi i k/5} \quad \text{and} \quad e^{2\pi i \ell/5}
    \]
    for some \(k,\ell \in \{0,1,2,3,4\}\). Their product is
    \[
    e^{2\pi i k/5} \cdot e^{2\pi i \ell/5}
    = e^{2\pi i (k + \ell) / 5}.
    \]
    Since \(k+\ell\) is an integer, \((k+\ell) \mod 5\) is again an integer in \(\{0,1,2,3,4\}\). Hence,
    \[
    e^{2\pi i (k+\ell)/5} \in \bigl\{ e^{2\pi i m/5} : m = 0,1,2,3,4 \bigr\}.
    \]
    Thus, the product is in \(G\), proving \textit{closure}.
    
    \item \textbf{Associativity:}\\
    Multiplication of complex numbers in \(\mathbb{C}\) is associative. Hence, for any \(a,b,c \in G\),
    \[
    (a \cdot b) \cdot c = a \cdot (b \cdot c).
    \]
    This proves \textit{associativity}.
    
    \item \textbf{Identity element:}\\
    We need \(e_G \in G\) such that \(e_G \cdot a = a\) for every \(a \in G\). Note that
    \[
    1 = e^{2\pi i \cdot 0/5} \in G,
    \]
    and for any \(e^{2\pi i k/5} \in G\),
    \[
    1 \cdot e^{2\pi i k/5} = e^{2\pi i k/5},
    \quad
    e^{2\pi i k/5} \cdot 1 = e^{2\pi i k/5}.
    \]
    Hence, \(1\) is the \textit{identity element} of \(G\).
    
    \item \textbf{Inverse element:}\\
    For \(a = e^{2\pi i k/5} \in G\), we claim the inverse is \(e^{-2\pi i k/5}\). Indeed,
    \[
    e^{2\pi i k/5} \cdot e^{-2\pi i k/5}
    = e^{2\pi i (k - k)/5}
    = e^0 = 1.
    \]
    Also, \(-k \equiv (5-k) \pmod{5}\), so \(e^{-2\pi i k/5} \in G\). Thus every element has an \textit{inverse} in \(G\).
    
    \item \textbf{Commutativity:}\\
    For \(a = e^{2\pi i k/5}, b = e^{2\pi i \ell/5} \in G\),
    \[
    a \cdot b = e^{2\pi i (k + \ell)/5}
    = e^{2\pi i (\ell + k)/5}
    = b \cdot a.
    \]
    Hence multiplication is commutative.
    
    \end{enumerate}

As a result, $(G,\cdot)$ is an Abelian group.\\

(b)

Listing all the elements in $G$, we get:

\begin{itemize}
\item $e^0 = 1$
\item $e^{2\pi i/5}$
\item $e^{4\pi i/5}$
\item $e^{6\pi i/5}$
\item $e^{8\pi i/5}$
\end{itemize}

\begin{enumerate}[(i)]
\item First we consider the trivial set $\{e^0 = 1\}$. This is a subgroup of $G$ because it is closed under the operation of $G$: $1 \cdot 1 = 1$, and for every element in it, its inverse is also in it so that $1 \cdot 1^{-1} = 1$.

\item Next we consider any proper subset $H$ of $G$ that contains any elements other than the identity element $e^0 = 1$, i.e., $a = e^{2\pi i k/5} \text{, for some } k \in \{1,2,3,4\}$.

According to the definition of a subgroup, $H$ must be closed under the operation of $G$. Therefore, for any $a \in H$, $a \cdot a \in H$, meaning that $a^2 = e^{4\pi i k/5} \in H$.

And since $a^2 \in H$, $a \cdot a^2 \in H$, meaning that $a^3 = e^{6\pi i k/5} \in H$.

And since $a^3 \in H$, $a \cdot a^3 \in H$, meaning that $a^4 = e^{8\pi i k/5} \in H$.

So far, we have $1, a, a^2, a^3, a^4 \in H$, which means $\{1, e^{2\pi i k/5}, e^{4\pi i k/5}, e^{6\pi i k/5}, e^{8\pi i k/5}\} \subseteq H$.

Let $k=1$, we get $\{1, e^{2\pi i/5}, e^{4\pi i/5}, e^{6\pi i/5}, e^{8\pi i/5}\} = G \subseteq H$. However, $H$ is a subset of $G$, so $H=G$.

\end{enumerate}

Hence, the only subgroups of $G$ are $\{1\}, G$.

\newpage

\section*{Exercise 4}

Let $d(x,y)$ stand for the (Euclidean) distance between point $x,y \in \mathbb{R}^2$. An isometry of $\mathbb{R}^2$ is a bijective map $T: \mathbb{R}^2 \to \mathbb{R}^2$ satisfying $d(Tx,Ty)=d(x,y)$ for all $x,y \in \mathbb{R}^2$. Show that the set of isometries of $\mathbb{R}^2$ forms a group with respect to the operation of composition (that is, $T_1 \circ T_2$ is the map defined by $(T_1 \circ T_2)(x)=T_1(T_2(x))$ for $x \in \mathbb{R}^2$).

\textbf{Solution:}



\end{document}