\documentclass{article}
\usepackage{graphicx}
\usepackage{amsmath}
\usepackage{array}
\usepackage{fancyhdr}
\usepackage{amssymb}
\usepackage[shortlabels]{enumitem}

\pagestyle{fancy}
\fancyhead[L]{Banghao Chi}
\fancyhead[C]{Homework 1}
\fancyhead[R]{25th Jan}

\fancyfoot[C]{\thepage}

\renewcommand{\headrulewidth}{0.5pt}
\renewcommand{\footrulewidth}{0.5pt}

\begin{document}

\section*{Exercise 1}

Consider the group $(G,\star)$, where $G=\{-1,0,1\}$ and the group operation is defined by
\[
a \star b = \begin{cases}
a + b + 1, & \text{if } a + b \leq 0, \\
a + b - 2, & \text{else.}
\end{cases}
\]

(a) Write the following multiplication table for this group (in the row corresponding to an element $a \in G$ and the column corresponding to an element $b \in G$ one should write the element $a \star b$):
\[
\begin{array}{c|ccc}
& -1 & 0 & 1 \\
\hline
-1 & & & \\
0 & & & \\
1 & & & 
\end{array}
\]

(b) What is the identity element for this group?

(c) What is the inverse operation?

(d) Verify an example of the associativity property of $G$.\\

\textbf{Solution:}\\

(a)

\[
\begin{array}{c|ccc}
& -1 & 0 & 1 \\
\hline
-1 & -1 & 0 & 1\\
0 & 0 & 1 & -1\\
1 & 1 & -1 & 0
\end{array}
\]

(b) From the above table, the identity element is -1. \\

(c) In order to find the inverse operation, we need to find formulas that give us identity element($a \star a^{-1} = a^{-1} \star a = e$), which in this case $e = -1$.

From the table, we can see that:

\begin{align*}
-1 \star -1 &= -1 \\
0 \star 1 &= -1 \\
1 \star 0 &= -1
\end{align*}

Therefore, the inverse function is:
\[
a^{-1} = \begin{cases}
-1, & \text{if } a = -1, \\
1, & \text{if } a = 0, \\
0, & \text{if } a = 1.
\end{cases}
\]

(d) For example:
\begin{align*}
(-1 \star 0) \star 1 &= 0 \star 1 = -1 \\
-1 \star (0 \star 1) &= -1 \star -1 = -1
\end{align*}

Therefore, the associativity property is satisfied.

\newpage

\section*{Exercise 2}

Let $G$ be a group.
\begin{enumerate}[(a)]
\item Let $a,b \in G$. Show that there exists a unique element $x \in G$ such that $ax=b$.
\item Prove the cancellation law: if $ab=ac$ for some $a,b,c \in G$, then $b=c$.
\end{enumerate}

\textbf{Solution:}\\

(a) Since $G$ is a group, there exists its inverse element $a^{-1} \in G$ such that $a \star a^{-1} = a^{-1} \star a = e$. Therefore, we can construct the x by:

$$x = a^{-1} \star b$$

Now, we will prove the existence and the uniqueness of x.

\begin{enumerate}[(i)]
\item Existence:

In order to verify the existence of x, we need to plug it into the equation $ax=b$ and see if the equation holds.

$$a \star x = a \star (a^{-1} \star b) = (a \star a^{-1}) \star b = e \star b = b$$

Therefore, x exists.

\item Uniqueness:

In order to verify the uniqueness of x, we need to prove that if there exists another element $x' \in G$ such that $ax'=b$, then $x=x'$.

\begin{align*}
x & = a^{-1} \star b \\
& = a^{-1} \star (a \star x') \\
& = (a^{-1} \star a) \star x' \\
& = e \star x' \\
& = x'
\end{align*}

Therefore, x is unique.

\end{enumerate}

\newpage

\section*{Exercise 3}

Consider the set $G=\{e^{2\pi i k/5} : k=0,1,2,3,4\}$ with the operation of usual multiplication $a \cdot b := ab$.
\begin{enumerate}[(a)]
\item Show that $(G,\cdot)$ is an Abelian group.
\item What subgroups does this group have?
\end{enumerate}

\textbf{Solution:}\\

(a) In order to show that $(G,\cdot)$ is an Abelian group, we need to show that it is a group(closed, associative, identity, inverse) and that it is Abelian(commutative).

\begin{enumerate}[(i)]
\item Closure: When we multiply two elements $e^{2\pi i k_1/5}$ and $e^{2\pi i k_2/5}$:
$e^{2\pi i k_1/5} \cdot e^{2\pi i k_2/5} = e^{2\pi i(k_1 + k_2)/5}$.
For $k_1, k_2 \in {0,1,2,3,4}$, we get:
$k_1 + k_2$ could be 0,1,2,3,4,5,6,7,8.
For the sums that exceed 4, note that:
$e^{2\pi i} = 1$, so:
$e^{2\pi i(k_1 + k_2)/5} = e^{2\pi i(k_1 + k_2)/5} \cdot (e^{2\pi i})^{-n}$ where n is chosen so that $(k_1 + k_2 - 5n)$ is in ${0,1,2,3,4}$.

\item Associativity:

\item Identity element:

\item Inverse element:

\item Commutativity:

\end{enumerate}

\newpage

\section*{Exercise 4}

Let $d(x,y)$ stand for the (Euclidean) distance between point $x,y \in \mathbb{R}^2$. An isometry of $\mathbb{R}^2$ is a bijective map $T: \mathbb{R}^2 \to \mathbb{R}^2$ satisfying $d(Tx,Ty)=d(x,y)$ for all $x,y \in \mathbb{R}^2$. Show that the set of isometries of $\mathbb{R}^2$ forms a group with respect to the operation of composition (that is, $T_1 \circ T_2$ is the map defined by $(T_1 \circ T_2)(x)=T_1(T_2(x))$ for $x \in \mathbb{R}^2$).

\textbf{Solution:}

\newpage

\section*{Exercise 5 (Optional)}

Show that for any subgroup $H$ of the group $(\mathbb{Z},+)$ there is $k \in \mathbb{Z}$ such that $H=\{ka : a \in \mathbb{Z}\}$ (here $ka$ stands for the usual product of numbers $k$ and $a$).

\textbf{Solution:}

\end{document}