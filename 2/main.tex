\documentclass{article}
\usepackage{graphicx}
\usepackage{amsmath}
\usepackage{array}
\usepackage{fancyhdr}
\usepackage{amssymb}
\usepackage[shortlabels]{enumitem}

\pagestyle{fancy}
\fancyhead[L]{Banghao Chi}
\fancyhead[C]{Homework 2}
\fancyhead[R]{7th Feb}

\fancyfoot[C]{\thepage}

\renewcommand{\headrulewidth}{0.5pt}
\renewcommand{\footrulewidth}{0.5pt}

\begin{document}

\section*{Exercise 1}
Calculate the inverses for all elements in $\mathbb{Z}_{18}^*$. \\

\textbf{Solution:} \\

First, finding all elements in $\mathbb{Z}_{18}^*$ (numbers coprime to 18):

$$\mathbb{Z}_{18}^* = \{1, 5, 7, 11, 13, 17\}$$

For $a \in \mathbb{Z}_{18}^*$, we need to find $b$ such that $ab \equiv 1 \pmod{18}$ \\

1. For $a = 1$:
  $$1 \cdot 1 \equiv 1 \pmod{18} \therefore 1^{-1} = 1$$

2. For $a = 5$:
  $$5 \cdot 11 \equiv 55 \equiv 1 \pmod{18} \therefore 5^{-1} = 11$$

3. For $a = 7$:
  $$7 \cdot 13 \equiv 91 \equiv 1 \pmod{18} \therefore 7^{-1} = 13$$

4. For $a = 11$:
  $$11 \cdot 5 \equiv 55 \equiv 1 \pmod{18} \therefore 11^{-1} = 5$$

5. For $a = 13$:
  $$13 \cdot 7 \equiv 91 \equiv 1 \pmod{18} \therefore 13^{-1} = 7$$

6. For $a = 17$:
  $$17 \cdot 17 \equiv 289 \equiv 1 \pmod{18} \therefore 17^{-1} = 17$$

Therefore, the complete list of inverses for all elements in $\mathbb{Z}_{18}^*$ is:
\begin{align*}
1^{-1} &= 1 \\
5^{-1} &= 11 \\
7^{-1} &= 13 \\
11^{-1} &= 5 \\
13^{-1} &= 7 \\
17^{-1} &= 17
\end{align*}

\newpage

\section*{Exercise 2}
Calculate
\begin{itemize}
\item[(a)] $3^{100} \pmod{10}$
\item[(b)] $5^{60} \pmod{7}$
\item[(c)] $400^{60} \pmod{61}$
\item[(d)] the last digit of $17^{50}$
\end{itemize}

\textbf{Solution:}
\begin{itemize}
\item[(a)] To find $3^{100} \pmod{10}$:
    \begin{itemize}
        \item Note that $3^4 = 81 \equiv 1 \pmod{10}$
        \item Therefore, $3^{100} \equiv 3^{4 \times 25} \equiv 1^{25} \equiv 1 \pmod{10}$
    \end{itemize}

\item[(b)] To find $5^{60} \pmod{7}$:
    \begin{itemize}
        \item By Euler's theorem, $\phi(7) = 6$
        \item Therefore, $5^6 \equiv 1 \pmod{7}$
        \item Hence, $5^{60} \equiv 5^{6 \times 10} \equiv 1^{10} \equiv 1 \pmod{7}$
    \end{itemize}

\item[(c)] To find $400^{60} \pmod{61}$:
    \begin{itemize}
        \item Note that $400 \equiv 34 \pmod{61}$ (since $400 = 6 \times 61 + 34$)
        \item Therefore, $400^{60} \equiv 34^{60} \pmod{61}$
        \item By Euler's theorem, $\phi(61) = 60$
        \item $34^{60} \equiv 1 \pmod{61}$
        \item Therefore, $400^{60} \equiv 34^{60} \equiv 1 \pmod{61}$
    \end{itemize}

\item[(d)] To find the last digit of $17^{50}$:
    \begin{itemize}
        \item This is equivalent to finding $17^{50} \pmod{10}$
        \item $17 \equiv 7 \pmod{10}$
        \item Therefore, we need to find $7^{50} \pmod{10}$
        \item By Euler's theorem, $\phi(10) = 4$
        \item Therefore, $7^4 \equiv 1 \pmod{10}$
        \item Therefore, $17^{50} \equiv 7^{50} \equiv 7^{4 \times 12 + 2} \equiv 1^{12} \times 7^2 \equiv 7^2 \equiv 9 \pmod{10}$
    \end{itemize}

\end{itemize}

\newpage

\section*{Exercise 3}
Let $\varphi: G \to H$ be a group isomorphism. Show that the inverse map $\varphi^{-1}: H \to G$ is also an isomorphism. \\

\textbf{Solution:} \\

To show that $\varphi^{-1}$ is an isomorphism, we need to prove that it is bijective and is a homomorphism. \\

\textbf{1) } $\varphi^{-1}$ is a homomorphism: \\

Let $h_1, h_2 \in H$. We need to show that $\varphi^{-1}(h_1h_2) = \varphi^{-1}(h_1)\varphi^{-1}(h_2)$ \\

Since $\varphi$ is an isomorphism, there exist unique $g_1, g_2 \in G$ such that:
$\varphi(g_1) = h_1$ and $\varphi(g_2) = h_2$ \\

Therefore, $\varphi^{-1}(h_1) = g_1$ and $\varphi^{-1}(h_2) = g_2$ \\

On the other hand, since $\varphi$ is a homomorphism:
$\varphi(g_1g_2) = \varphi(g_1)\varphi(g_2) = h_1h_2$, we get:
$\varphi^{-1}(h_1h_2) = g_1g_2$ \\

Therefore:
$\varphi^{-1}(h_1h_2) = g_1g_2 = \varphi^{-1}(h_1)\varphi^{-1}(h_2)$ \\

This shows that $\varphi^{-1}$ is a homomorphism. \\

\textbf{2) } $\varphi^{-1}$ is bijective: \\

Since $\varphi$ is an isomorphism, it is bijective. By definition of inverse function:
\begin{itemize}
    \item $\varphi^{-1}$ is injective because $\varphi$ is surjective
    \item $\varphi^{-1}$ is surjective because $\varphi$ is injective
\end{itemize}

Therefore, $\varphi^{-1}$ is a bijective homomorphism, making it an isomorphism.

\newpage

\section*{Exercise 4}
Show that the multiplicative group $C_n = \{e^{2\pi i k/n} : k = 0,1,\ldots,n-1\}$ of $n$-th roots of unity is isomorphic to $\mathbb{Z}_n$.

\textbf{Solution:}
\newpage

\section*{Exercise 5}
Suppose that $\varphi: G \to H$ is a homomorphism, and let $e_G$ and $e_H$ be the identity elements of $G$ and $H$, respectively. Show that
\begin{itemize}
\item[(a)] for all $g \in G$ and $n \in \mathbb{N}$, we have $\varphi(g^n) = (\varphi(g))^n$
\item[(b)] if $g^n = e_G$, then $\varphi(g)^n = e_H$
\end{itemize}

\textbf{Solution:}
\newpage

\section*{Exercise 6}
Show that a group $G$ is abelian if and only if $(ab)^2 = a^2b^2$ for all $a,b \in G$.

\textbf{Solution:}

\end{document}