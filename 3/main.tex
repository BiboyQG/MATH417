\documentclass{article}
\usepackage{graphicx}
\usepackage{amsmath}
\usepackage{array}
\usepackage{fancyhdr}
\usepackage{amssymb}
\usepackage[shortlabels]{enumitem}

\pagestyle{fancy}
\fancyhead[L]{Banghao Chi}
\fancyhead[C]{Homework 3}
\fancyhead[R]{15th Feb}

\fancyfoot[C]{\thepage}

\renewcommand{\headrulewidth}{0.5pt}
\renewcommand{\footrulewidth}{0.5pt}

\begin{document}

\section*{Exercise 1}
Calculate the order of each element in $\mathbb{Z}_{18}^*$. \\

\textbf{Solution:} \\

From Homework 2, we know that

$$\mathbb{Z}_{18}^* = \{1, 5, 7, 11, 13, 17\}$$

1) For 1:
\begin{align*}
   1^1 = 1
\end{align*}

Therefore, the order of 1 is 1 \\

2) For 5:
\begin{align*}
   5^1 \equiv 5 \pmod{18} \\
   5^2 \equiv 7 \pmod{18} \\
   5^3 \equiv 17 \pmod{18} \\
   5^4 \equiv 13 \pmod{18} \\
   5^5 \equiv 11 \pmod{18} \\
   5^6 \equiv 1 \pmod{18}
\end{align*}

Therefore, the order of 5 is 6 \\

3) For 7:
\begin{align*}
   7^1 \equiv 7 \pmod{18} \\
   7^2 \equiv 13 \pmod{18} \\
   7^3 \equiv 1 \pmod{18}
\end{align*}

Therefore, the order of 7 is 3 \\

4) For 11:
\begin{align*}
   11^1 \equiv 11 \pmod{18} \\
   11^2 \equiv 13 \pmod{18} \\
   11^3 \equiv 17 \pmod{18} \\
   11^4 \equiv 7 \pmod{18} \\
   11^5 \equiv 5 \pmod{18} \\
   11^6 \equiv 1 \pmod{18}
\end{align*}

Therefore, the order of 11 is 6 \\

5) For 13:
\begin{align*}
   13^1 \equiv 13 \pmod{18} \\
   13^2 \equiv 7 \pmod{18} \\
   13^3 \equiv 1 \pmod{18}
\end{align*}

Therefore, the order of 13 is 3 \\

6) For 17:
\begin{align*}
   17^1 \equiv 17 \pmod{18} \\
   17^2 \equiv 1 \pmod{18}
\end{align*}

Therefore, the order of 17 is 2 \\

\textbf{Therefore:}
\begin{align*}
    ord(1) = 1 \\
    ord(5) = 6 \\
    ord(7) = 3 \\
    ord(11) = 6 \\
    ord(13) = 3 \\
    ord(17) = 2
\end{align*}

\newpage

\section*{Exercise 2}
Let $\psi: G \to K$ be a group isomorphism. Show that
\begin{itemize}
\item[(a)] if $G$ is abelian, then so is $K$;
\item[(b)] if $H \triangleleft G$, then $\psi(H) \triangleleft K$;
\item[(c)] $\text{ord}_G(a) = \text{ord}_K(\psi(a))$ for any $a \in G$.
\end{itemize}

\textbf{Solution:} \\

(a) Let $k_1, k_2 \in K$. Since $\psi$ is an isomorphism, there exist $g_1, g_2 \in G$ such that $\psi(g_1) = k_1$ and $\psi(g_2) = k_2$.
Then:
\begin{align*}
k_1k_2 &= \psi(g_1)\psi(g_2) \\
&= \psi(g_1g_2) \text{ (since $\psi$ is a homomorphism)} \\
&= \psi(g_2g_1) \text{ (since $G$ is abelian)} \\
&= \psi(g_2)\psi(g_1) \\
&= k_2k_1
\end{align*}

Therefore, $K$ is abelian. \\

(b) Let $k \in K$. We want to show that $k\psi(H)k^{-1} \subseteq \psi(H)$. \\

Since $\psi$ is an isomorphism, there exists $g \in G$ such that $\psi(g) = k$.
Let $h' \in \psi(H)$. Then there exists $h \in H$ such that $\psi(h) = h'$. \\

Consider:
\begin{align*}
kh'k^{-1} &= \psi(g)\psi(h)\psi(g^{-1}) \\
&= \psi(ghg^{-1}) \\
&\in \psi(H) \text{ (since $H \triangleleft G$ implies $ghg^{-1} \in H$)}
\end{align*}

Therefore, $\psi(H) \triangleleft K$. \\

(c) Let $n = \text{ord}_G(a)$. Then $a^n = e_G$ and $a^m \neq e_G$ for any $m < n$.
\begin{align*}
\psi(a^n) &= \psi(e_G) = e_K \\
\psi(a^n) &= (\psi(a))^n \text{ (by homomorphism property)}
\end{align*}

Therefore, $(\psi(a))^n = e_K$, meaning that $\text{ord}_K(\psi(a)) \leq n$. \\

Suppose there exists $m < n$ such that $(\psi(a))^m = e_K$.
Then $\psi(a^m) = e_K$. Since $\psi$ is injective, this implies $a^m = e_G$, contradicting the minimality of $n$. \\

Therefore, $\text{ord}_K(\psi(a)) = \text{ord}_G(a)$.

\newpage

\section*{Exercise 3}
Let $G$ be a finite group and $f: G \to K$ be a group homomorphism. Show that $|G| = |\text{Im } f| \cdot |\text{ker } f|$. \\

\textbf{Solution:} \\

By the First Isomorphism Theorem, we know that:
   $$G/\text{ker } f \cong \text{Im } f$$

Due to the isomorphism, we have:
\begin{align*}
   |G/\text{ker } f| &= |\text{Im } f| \\
   \frac{|G|}{|\text{ker } f|} &= |\text{Im } f|
\end{align*}

Hence:
\begin{align*}
   |G| &= |\text{Im } f| \cdot |\text{ker } f|
\end{align*}

\newpage

\section*{Exercise 4}
Show that
\begin{itemize}
\item[a)] $\text{GL}_n(\mathbb{R})/\text{SL}_n(\mathbb{R}) \simeq \mathbb{R}^\times$;
\item[b)] $\mathbb{C}^\times/C_n \simeq \mathbb{C}^\times$;
\item[c)] $\mathbb{C}/\mathbb{R} \simeq \mathbb{R}$.
\end{itemize}
(Here $\mathbb{C}$ and $\mathbb{R}$ stand for the additive groups of complex and real numbers, respectively, and $C_n = \{e^{2\pi i k/n}: k = 0,1,...,n-1\}$ stands for the multiplicative group of $n$-th roots of unity.) \\

\textbf{Solution:} \\

(a) To show $\text{GL}_n(\mathbb{R})/\text{SL}_n(\mathbb{R}) \simeq \mathbb{R}^\times$: \\

The determinant map $\det: \text{GL}_n(\mathbb{R}) \to \mathbb{R}^\times$ is a surjective group homomorphism. \\

By the First Isomorphism Theorem, we have:
$\text{GL}_n(\mathbb{R})/\ker(\det) \simeq \mathbb{R}^\times$ \\

By definition, $\ker(\det) = \text{SL}_n(\mathbb{R})$.
Therefore, $\text{GL}_n(\mathbb{R})/\text{SL}_n(\mathbb{R}) \simeq \mathbb{R}^\times$ \\

(b) To show $\mathbb{C}^\times/C_n \simeq \mathbb{C}^\times$: \\

The map $f: \mathbb{C}^\times \to \mathbb{C}^\times$ defined by $f(z) = z^n$ is a surjective group homomorphism. \\

The kernel of $f$ is precisely $C_n = \{e^{2\pi i k/n}: k = 0,1,...,n-1\}$
since these are exactly the elements that map to 1 under $f$. \\

By the First Isomorphism Theorem:
$\mathbb{C}^\times/\ker(f) \simeq \text{Im}(f) = \mathbb{C}^\times$ \\

Therefore, $\mathbb{C}^\times/C_n \simeq \mathbb{C}^\times$ \\

(c) To show $\mathbb{C}/\mathbb{R} \simeq \mathbb{R}$: \\

The map $g: \mathbb{C} \to \mathbb{R}$ defined by $g(z) = \text{Im}(z)$
where $\text{Im}(z)$ denotes the imaginary part of $z$. \\

This is a surjective group homomorphism with kernel $\mathbb{R}$
(since real numbers are precisely the complex numbers with zero imaginary part). \\

By the First Isomorphism Theorem:
$\mathbb{C}/\ker(g) \simeq \text{Im}(g) = \mathbb{R}$ \\

Also,
\begin{align*}
   z \in \ker(g) &\iff g(z) = 0 \\
   &\iff y = 0 \text{ (where $z = x + yi$)} \\
   &\iff z = x \text{ for some $x \in \mathbb{R}$} \\
   &\iff z \in \mathbb{R}
\end{align*}

meaning that $\ker(g) = \mathbb{R}$ \\

Therefore, $\mathbb{C}/\mathbb{R} \simeq \mathbb{R}$

\newpage

\section*{Exercise 5}
For each pair of groups from the following list, determine whether they are isomorphic or not:
\[ \mathbb{Z}_{12}, \quad \mathbb{Z}_3 \oplus \mathbb{Z}_2 \oplus \mathbb{Z}_2, \quad \mathbb{Z}_2 \oplus \mathbb{Z}_3 \oplus \mathbb{Z}_4, \quad \mathbb{Z}_3 \oplus \mathbb{Z}_4, \quad C_{12}, \quad C_4 \times C_3. \]

\textbf{Solution:} \\



\newpage

\section*{Exercise 6}
Let $G$ be a group such that any non-identity element has order 2. Show that $G$ is abelian. \\

\textbf{Solution:} \\



\end{document}